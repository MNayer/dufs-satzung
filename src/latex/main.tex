\documentclass[a4paper]{article}

\usepackage[T1]{fontenc}
\usepackage[utf8]{inputenc}
\usepackage[german]{babel}

\usepackage{multicol}

\setlength{\parindent}{0em}
\setlength{\parskip}{1em}

% Paragraph Symbole für Sections
\usepackage{titlesec}
\titleformat{\section}
  {\normalfont\Large\bfseries}{\S\thesection}{1em}{}

% Korrekte Enumerations
%\renewcommand{\labelenumi}{(\theenumi)}
%\renewcommand{\labelenumii}{\theenumii)}
%\renewcommand{\labelenumiii}{\theenumiii}
%\renewcommand{\labelenumiv}{\theenumiv}

% Parametrisierung des Aufzählungsstils
\usepackage{enumitem}


\title{Vereinssatzung\\Datenschutz und Freie Software}

\begin{document}
\maketitle

\section{Name und Sitz des Vereins, Geschäftsjahr}
\begin{enumerate}[label=(\arabic*)]
    \item Der Verein führt den Namen ``Datenschutz und Freie Software``. Er soll
        in das Vereinsregister eingetragen werden und führt danach den Zusatz
        ``e.V.``.
    \item Der Verein hat seinen Sitz in Münster.
    \item Das Geschäftsjahr ist das Kalenderjahr.
\end{enumerate}

\section{Zweck, Gemeinnützigkeit des Vereins}
\begin{enumerate}[label=(\arabic*)]
    \item Der Verein mit Sitz in Münster verfolgt ausschließlich und
        unmittelbar gemeinnützige Zwecke im Sinne des Abschnitts
        ``Steuerbegünstigte Zwecke`` der Abgabenordnung.
    \item Der Zweck des Vereins ist die Förderung
        \begin{enumerate}[label=\alph*)]
            \item von Verbraucherberatung und Verbraucherschutz
            \item von Wissenschaft und Forschung 
        \end{enumerate}
        Der Satzungszweck nach Absatz 2 lit. a) wird insbesondere verwirklicht
        durch
        \begin{enumerate}[label=\alph*)]
            \item die Entwicklung von Systemen zur Unterstützung von
                Verbrauchern bei der Wahrnehmung ihrer Datenschutz-Rechte
            \item die Beratung und Aufklärung über Datenschutz-Rechte und
                Pflichten im Umgang mit personenbezogenen Daten
        \end{enumerate}
        Der Satzungszweck nach Absatz 2 lit. b) wird insbesondere verwirklicht
        durch
        \begin{enumerate}[label=\alph*)]
            \item die Durchführung, Auswertung und Veröffentlichung von
                Projekten wissenschaftlicher Art zur Untersuchung oder
                Umsetzung von Systemen, die freie Software oder den Schutz von
                Daten fördern
        \end{enumerate}
    \item Der Verein ist selbstlos tätig; er verfolgt nicht in erster Linie
        eigenwirtschaftliche Zwecke.
    \item Mittel des Vereins dürfen nur für die satzungsmäßigen Zwecke
        verwendet werden. Die Mitglieder erhalten keine Zuwendungen aus den
        Mitteln des Vereins.
    \item Es darf keine Person durch Ausgaben, die dem Zweck des Vereins fremd
        sind, oder durch unverhältnismäßig hohe Vergütungen begünstigt werden.
\end{enumerate}

\section{Erwerb der Mitgliedschaft}
\begin{enumerate}[label=(\arabic*)]
    \item Mitglied des Vereins kann jede natürliche Person werden.
    \item Der Aufnahmeantrag ist in Textform unter Angabe einer postalischen und einer 
        e-Mail Adresse beim Vorstand zu beantragen. Der Vorstand entscheidet über den
        Aufnahmeantrag nach freiem Ermessen. Eine Ablehnung des Antrags muss
        er gegenüber dem Antragsteller nicht begründen.
    \item Auf Vorschlag des Vorstands kann die Mitgliederversammlung Mitglieder
        oder sonstige Personen, die sich um den Verein besonders verdient
        gemacht haben, zu Ehrenmitgliedern auf Lebenszeit ernennen.
\end{enumerate}

\section{Beendigung der Mitgliedschaft}
\begin{enumerate}[label=(\arabic*)]
    \item Die Mitgliedschaft im Verein endet durch Tod, Austritt oder Ausschluss.
    \item Der Austritt ist schriftlich gegenüber dem Vorstand zu erklären. Der
        Austritt kann jederzeit erklärt werden.
    \item Ein Mitglied kann durch Beschluss der Mitgliederversammlung aus dem
        Verein ausgeschlossen werden, wenn es schuldhaft das Ansehen oder die
        Interessen des Vereins in schwerwiegender Weise schädigt.
\end{enumerate}

\section{Rechte und Pflichten der Mitglieder}
\begin{enumerate}[label=(\arabic*)]
    \item Jedes Mitglied hat das Recht an gemeinsamen Veranstaltungen
        teilzunehmen. Jedes Mitglied hat gleiches Stimm- und Wahlrecht in der
        Mitgliederversammlung.
    \item Jedes Mitglied hat die Pflicht, die Interessen des Vereins zu fördern
        und, soweit es in seinen Kräften steht, das Vereinsleben durch seine
        Mitarbeit zu unterstützen.
\end{enumerate}

\section{Aufnahmegebühr und Mitgliedsbeiträge}
Von den Mitgliedern werden keine Mitgliedsbeiträge erhoben.
%\begin{enumerate}[label=(\arabic*)] \item Jedes Mitglied hat einen im Voraus fällig werdenden monatlichen
%        Mitgliedsbeitrag zu entrichten.
%    \item Die Höhe der Aufnahmegebühr und der Mitgliedsbeiträge wird von der
%        Mitgliederversammlung festgelegt.
%    \item Ehrenmitglieder sind von der Aufnahmegebühr und den
%        Mitgliedsbeiträgen befreit.
%\end{enumerate}

\section{Organe des Vereins}
Organe des Vereins sind der Vorstand und die Mitgliederversammlung.

\section{Vorstand}
\begin{enumerate}[label=(\arabic*)]
    \item Der Vorstand besteht aus bis zu drei Vereinsmitgliedern.
    \item Ist nur ein Vereinsmitglied zum Vorstand bestellt, ist dieses stets
        einzelvertretungsberechtigt.
    \item Besteht der Vorstand aus mehreren Vereinsmitgliedern, wird der Verein
        durch zwei Vorstandsmitglieder gemeinschaftlich vertreten.
    \item Dem Vorstand wird keine Vergütung gezahlt.
\end{enumerate}

\section{Aufgaben des Vorstands}
Dem Vorstand des Vereins obliegen die Vertretung des Vereins nach § 26 BGB und
die Führung seiner Geschäfte. Er hat insbesondere folgende Aufgaben:
\begin{enumerate}[label=\alph*)]
    \item die Einberufung und Vorbereitung der Mitgliederversammlungen
        einschließlich der Aufstellung der Tagesordnung,
    \item die Ausführung von Beschlüssen der Mitgliederversammlung,
    \item die Verwaltung des Vereinsvermögens und die Anfertigung des Jahresberichts,
    \item die Aufnahme neuer Mitglieder.
\end{enumerate}

\section{Bestellung des Vorstands}
\begin{enumerate}[label=(\arabic*)]
    \item Die Mitglieder des Vorstands werden von der Mitgliederversammlung für
        die Dauer von zwei Jahren einzeln gewählt. Mitglieder des Vorstands
        können nur Mitglieder des Vereins sein; mit der Mitgliedschaft im
        Verein endet auch die Mitgliedschaft im Vorstand.
    \item Die Wiederwahl oder die vorzeitige Abberufung eines Mitglieds durch
        die Mitgliederversammlung ist zulässig. Ein Mitglied bleibt nach Ablauf
        der regulären Amtszeit bis zur Wahl seines Nachfolgers im Amt.
\end{enumerate}

\section{Beratung und Beschlussfassung des Vorstands}
\begin{enumerate}[label=(\arabic*)]
    \item Der Vorstand tritt nach Bedarf zusammen. Die Sitzungen werden vom
        Vorsitzenden einberufen. Der Vorstand ist beschlussfähig, wenn der
        Vorsitzende anwesend ist. Bei der Beschlussfassung entscheidet der
        der Vorsitzende.
    \item Die Beschlüsse des Vorstands sind zu protokollieren. Das Protokoll
        ist vom Vorsitzenden zu unterschreiben.
\end{enumerate}

\section{Aufgaben der Mitgliederversammlung}
Die Mitgliederversammlung ist zuständig für die Entscheidungen in folgenden
Angelegenheiten:
\begin{enumerate}[label=\alph*)]
    \item Änderungen der Satzung,
    \item die Ernennung von Ehrenmitgliedern sowie der Ausschluss von
        Mitgliedern aus dem Verein,
    \item die Wahl und die Abberufung der Mitglieder des Vorstands,
    \item die Entgegennahme des Jahresberichts und die Entlastung des Vorstands,
    \item die Auflösung des Vereins.
\end{enumerate}

\section{Einberufung der Mitgliederversammlung}
\begin{enumerate}[label=(\arabic*)]
    \item Der Vorstand ist zur Einberufung der einer Mitgliederversammlung
        zumindest alle zwei Jahre verpflichtet. Die Einberufung erfolgt
        per e-Mail unter Einhaltung einer Frist von zwei Wochen und unter
        Angabe der Tagesordnung.
    \item Die Tagesordnung setzt der Vorstand fest. Jedes Vereinsmitglied kann
        bis spätestens eine Woche vor der Mitgliederversammlung beim Vorstand
        schriftlich eine Ergänzung der Tagesordnung beantragen. Über den Antrag
        entscheidet der Vorstand. Über Anträge zur Tagesordnung, die vom
        Vorstand nicht aufgenommen wurden oder die erstmals in der
        Mitgliederversammlung gestellt werden, entscheidet die
        Mitgliederversammlung mit der Mehrheit der Stimmen der anwesenden
        Mitglieder; dies gilt nicht für Anträge, die eine Änderung der Satzung,
        Änderungen der Mitgliedsbeiträge oder die Auflösung des Vereins zum
        Gegenstand haben.
    \item Der Vorstand hat eine außerordentliche Mitgliederversammlung
        einzuberufen, wenn es das Interesse des Vereins erfordert oder wenn
        mindestens ein Drittel der Mitglieder dies schriftlich unter Angabe des
        Zwecks und der Gründe beantragt.
\end{enumerate}

\section{Beschlussfassung der Mitgliederversammlung}
\begin{enumerate}[label=(\arabic*)]
    \item Die Mitgliederversammlung wird vom Vorsitzenden des Vorstands, bei
        dessen Verhinderung von einem durch die Mitgliederversammlung zu wählenden
        Versammlungsleiter geleitet.
    \item Die Mitgliederversammlung ist beschlussfähig, wenn mindestens ein
        Drittel aller Vereinsmitglieder anwesend ist. Bei Beschlussunfähigkeit
        ist der Vorstand verpflichtet, innerhalb von vier Wochen eine zweite
        Mitgliederversammlung mit der gleichen Tagesordnung einzuberufen.
        Diese ist ohne Rücksicht auf die Zahl der erschienenen Mitglieder
        beschlussfähig. Hierauf ist in der Einladung hinzuweisen.
    \item Die Mitgliederversammlung beschließt in offener Abstimmung mit der
        Mehrheit der Stimmen der anwesenden Mitglieder.
    \item Bei Vorstandswahlen ist gewählt, wer die Mehrheit der abgegebenen
        Stimmen auf sich vereinigt. Wird im ersten Wahlgang diese Mehrheit
        nicht erreicht, findet zwischen den beiden Kandidaten, die im ersten
        Wahlgang die meisten Stimmen auf sich vereinigt haben, eine Stichwahl
        statt. 
    \item Über den Ablauf der Mitgliederversammlung und die gefassten
        Beschlüsse ist ein Protokoll zu fertigen, das vom Versammlungsleiter zu
        unterschreiben ist.
\end{enumerate}

\section{Auflösung des Vereins, Beendigung aus anderen Gründen, Wegfall steuerbegünstigter Zwecke}
\begin{enumerate}[label=(\arabic*)]
    \item Im Falle der Auflösung des Vereins ist der Vorsitzende des Vorstands
        ein vertretungsberechtigter Liquidator, falls die
        Mitgliederversammlung keine anderen Personen beruft.
    \item Bei Auflösung oder Aufhebung des Vereins oder bei Wegfall
        steuerbegünstigter Zwecke fällt das Vermögen des Vereins an eine
        juristische Person des öffentlichen Rechts oder eine andere
        steuerbegünstigte Körperschaft, zwecks Verwendung für die Förderung von
        Verbraucherberatung und Verbraucherschutz oder die Förderung von
        Wissenschaft und Forschung.
    \item Die vorstehenden Bestimmungen gelten entsprechend, wenn dem Verein
        die Rechtsfähigkeit entzogen wurde.
\end{enumerate}

\section*{}
Münster, den \today

%Die Gründungsmitglieder

\vspace{2cm}

\begin{multicols}{2}

\begin{center}

Felix Weißberg

\vspace{1.5cm}

Till Sippel

\vspace{1.5cm}

Gerrit Orlowski

\vspace{0.85cm}

Moritz Voß

\vspace{0.85cm}

Jens Peitzmeier

\vspace{0.85cm}

Nabiel Yusif

\vspace{0.85cm}

Nico Silvestre Andrade
\end{center}
\end{multicols}
    
\end{document}

